\documentclass[a4paper,11pt]{report}
\usepackage[T1]{fontenc}
\usepackage[utf8]{inputenc}
\usepackage{lmodern}
\usepackage{titlesec}
\usepackage{exptech}
\usepackage{listings}
\usepackage{hyperref}
\usepackage{color}

% Pour les figures
\usepackage{pstricks}
\usepackage{epsfig}
\usepackage[justification=centering]{caption}
\usepackage{tikz}

\setlength{\parskip}{1em}
% Random code from stackoverflow refining chapters
% See http://tex.stackexchange.com/questions/110840/how-to-remove-chapter-numbering-without-removing-it-from-tableofcontents
\titleformat{\chapter}
  {\Large\bfseries} % format
  {}                % label
  {0pt}             % sep
  {\huge}         % before-code

%% ** Begin document ** %%

\title{Étude pratique : Amélioration de la complétion automatique de \LaTeX{}ila}
\author{Axel Caro\and François Boschet\and Maximilien Richer}
\date{2014-2015}

\begin{document}

% Configuration pour le code
\include{./lst_config}

\maketitle %affichage du titre

\chapter{Introduction}
\label{cha:Introduction}
Ce document contient la document utilisateur de notre mécanisme de complétion dynamique de la commande \textbf{\textbackslash{}ref} que nous avons ajouté au logiciel \LaTeX{}ila.
Pour plus de détails sur le projet, veuillez vous référer au rapport complet.

\chapter{Installation de \LaTeX{}ila}
\label{cha:Installation}

\section{Dépôts officiels}
\label{sec:depots}

\LaTeX{}ila est disponible au téléchargement dans les dépôts officiels de plusieurs distribution comme :
\begin{itemize}
\item Arch Linux
\item Debian
\item Fedora
\item Gentoo
\item Ubuntu
\end{itemize}

\section{Sources}
\label{sec:sources}

Vous pouvez aussi compiler \LaTeX{}ila via les sources. Soit en les téléchargeant sur le serveur ftp de gnome\footnote{\url{http://ftp.gnome.org/pub/GNOME/sources/latexila/}} soit en clonant le dépot git\footnote{\url{https://git.gnome.org/browse/latexila}}.

Vous pouvez également suivre le guide que nous avons réalisé qui explique comment installer la dernière version de Latexila en utilisant l'utilitaire Jhbuild disponible sur notre dépot git\footnote{\url{https://github.com/halfa/alambic/blob/master/report/howtoContribFromScratch.md}}

\chapter{Utilisation de la complétion}

\LaTeX{}ila propose une complétion automatique des proncipales commandes \LaTeX{}. Avec notre projet, le logiciel propose aussi la complétion pour la commande \textbf{\textbackslash{}ref}.

Le fonctionnement est très simple. Vous créé un label, via la commande \textbf{\textbackslash{}label}, dans un fichier, et l'invite de complétion vous proposera ce label pour compléter les arguments de la commande \textbf{\textbackslash{}ref}. 


\end{document}