\documentclass[a4paper,11pt]{report}
\usepackage[T1]{fontenc}
\usepackage[utf8]{inputenc}
\usepackage{lmodern}
\usepackage{titlesec}
\usepackage{exptech}
\usepackage{listings}
\usepackage{hyperref}
\usepackage{color}

% Pour les figures
\usepackage{pstricks}
\usepackage{epsfig}
\usepackage[justification=centering]{caption}
\usepackage{tikz}

\setlength{\parskip}{1em}
% Random code from stackoverflow refining chapters
% See http://tex.stackexchange.com/questions/110840/how-to-remove-chapter-numbering-without-removing-it-from-tableofcontents
\titleformat{\chapter}
  {\Large\bfseries} % format
  {}                % label
  {0pt}             % sep
  {\huge}         % before-code

%% ** Begin document ** %%

\title{Étude pratique : Amélioration de la complétion automatique de \LaTeX{}ila}
\author{Axel Caro\and François Boschet\and Maximilien Richer}
\date{2014-2015}

\begin{document}

% Configuration pour le code
\definecolor{mygreen}{rgb}{0,0.6,0}
\definecolor{mygray}{rgb}{0.5,0.5,0.5}
\definecolor{mymauve}{rgb}{0.58,0,0.82}

\lstset{ %
  backgroundcolor=\color{white},   % choose the background color; you must add \usepackage{color} or \usepackage{xcolor}
  basicstyle=\tt\small,       % the size of the fonts that are used for the code
  breakatwhitespace=false,         % sets if automatic breaks should only happen at whitespace
  breaklines=true,                 % sets automatic line breaking
  captionpos=b,                    % sets the caption-position to bottom
  commentstyle=\color{mygreen},    % comment style
  deletekeywords={...},            % if you want to delete keywords from the given language
  escapeinside={\%*}{*)},          % if you want to add LaTeX within your code
  extendedchars=true,              % lets you use non-ASCII characters; for 8-bits encodings only, does not work with UTF-8
  frame=single,                    % adds a frame around the code
  keepspaces=true,                 % keeps spaces in text, useful for keeping indentation of code (possibly needs columns=flexible)
  keywordstyle=\color{blue},       % keyword style
  language=[Sharp]C,                % the language of the code
  otherkeywords={*,...},            % if you want to add more keywords to the set
  numbers=left,                     % where to put the line-numbers; possible values are (none, left, right)
  numbersep=5pt,                    % how far the line-numbers are from the code
  numberstyle=\tiny\color{mygray}, % the style that is used for the line-numbers
  rulecolor=\color{black},          % if not set, the frame-color may be changed on line-breaks within not-black text (e.g. comments (green here))
  showspaces=false,                % show spaces everywhere adding particular underscores; it overrides 'showstringspaces'
  showstringspaces=false,          % underline spaces within strings only
  showtabs=false,                  % show tabs within strings adding particular underscores
  stepnumber=2,                    % the step between two line-numbers. If it's 1, each line will be numbered
  stringstyle=\color{mymauve},     % string literal style
  tabsize=2,                       % sets default tabsize to 2 spaces
  title=\lstname                   % show the filename of files included with \lstinputlisting; also try caption instead of title
}

\lstset{literate=
  {é}{{\'e}}1 {à}{{\`a}}1 {è}{{\`e}}1
 {â}{{\^a}}1 {ê}{{\^e}}1 {ç}{{\c c}}1
}


\maketitle %affichage du titre

\chapter{Introduction}
\label{cha:Introduction}
Ce document contient la document utilisateur de notre mécanisme de complétion dynamique de la commande \textbf{\textbackslash{}ref} que nous avons ajouté au logiciel \LaTeX{}ila.
Pour plus de détails sur le projet, veuillez vous référer au rapport complet.

\chapter{Installation de \LaTeX{}ila}
\label{cha:Installation}

\section{Dépôts officiels}
\label{sec:depots}

\LaTeX{}ila est disponible au téléchargement dans les dépôts officiels de plusieurs distribution comme :
\begin{itemize}
\item Arch Linux
\item Debian
\item Fedora
\item Gentoo
\item Ubuntu
\end{itemize}

\section{Sources}
\label{sec:sources}

Vous pouvez aussi compiler \LaTeX{}ila via les sources. Soit en les téléchargeant sur le serveur ftp de gnome\footnote{\url{http://ftp.gnome.org/pub/GNOME/sources/latexila/}} soit en clonant le dépot git\footnote{\url{https://git.gnome.org/browse/latexila}}.

Vous pouvez également suivre le guide que nous avons réalisé qui explique comment installer la dernière version de Latexila en utilisant l'utilitaire Jhbuild disponible sur notre dépot git\footnote{\url{https://github.com/halfa/alambic/blob/master/report/howtoContribFromScratch.md}}

\chapter{Utilisation de la complétion}

\LaTeX{}ila propose une complétion automatique des proncipales commandes \LaTeX{}. Avec notre projet, le logiciel propose aussi la complétion pour la commande \textbf{\textbackslash{}ref}.

Le fonctionnement est très simple. Vous créé un label, via la commande \textbf{\textbackslash{}label}, dans un fichier, et l'invite de complétion vous proposera ce label pour compléter les arguments de la commande \textbf{\textbackslash{}ref}. 


\end{document}