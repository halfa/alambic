\documentclass[a4paper,11pt]{report}
\usepackage[T1]{fontenc}
\usepackage[utf8]{inputenc}
\usepackage{lmodern}
\usepackage{titlesec}
\usepackage{exptech}
\usepackage{listings}
\usepackage{hyperref}
\usepackage{color}

% Pour les figures
\usepackage{pstricks}
\usepackage{epsfig}

\usepackage{tikz}

\setlength{\parskip}{1em}


%% ** Begin document ** %%

\title{Amélioration de la complétion automatique de \LaTeX{}ila : Documentation Technique}
\author{Axel Caro\and François Bochet\and Maximilien Richer}
\date{2014-2015}

\begin{document}

% Configuration pour le code
\definecolor{mygreen}{rgb}{0,0.6,0}
\definecolor{mygray}{rgb}{0.5,0.5,0.5}
\definecolor{mymauve}{rgb}{0.58,0,0.82}

\lstset{ %
  backgroundcolor=\color{white},   % choose the background color; you must add \usepackage{color} or \usepackage{xcolor}
  basicstyle=\tt\small,       % the size of the fonts that are used for the code
  breakatwhitespace=false,         % sets if automatic breaks should only happen at whitespace
  breaklines=true,                 % sets automatic line breaking
  captionpos=b,                    % sets the caption-position to bottom
  commentstyle=\color{mygreen},    % comment style
  deletekeywords={...},            % if you want to delete keywords from the given language
  escapeinside={\%*}{*)},          % if you want to add LaTeX within your code
  extendedchars=true,              % lets you use non-ASCII characters; for 8-bits encodings only, does not work with UTF-8
  frame=single,                    % adds a frame around the code
  keepspaces=true,                 % keeps spaces in text, useful for keeping indentation of code (possibly needs columns=flexible)
  keywordstyle=\color{blue},       % keyword style
  language=[Sharp]C,                % the language of the code
  otherkeywords={*,...},            % if you want to add more keywords to the set
  numbers=left,                     % where to put the line-numbers; possible values are (none, left, right)
  numbersep=5pt,                    % how far the line-numbers are from the code
  numberstyle=\tiny\color{mygray}, % the style that is used for the line-numbers
  rulecolor=\color{black},          % if not set, the frame-color may be changed on line-breaks within not-black text (e.g. comments (green here))
  showspaces=false,                % show spaces everywhere adding particular underscores; it overrides 'showstringspaces'
  showstringspaces=false,          % underline spaces within strings only
  showtabs=false,                  % show tabs within strings adding particular underscores
  stepnumber=2,                    % the step between two line-numbers. If it's 1, each line will be numbered
  stringstyle=\color{mymauve},     % string literal style
  tabsize=2,                       % sets default tabsize to 2 spaces
  title=\lstname                   % show the filename of files included with \lstinputlisting; also try caption instead of title
}

\lstset{literate=
  {é}{{\'e}}1 {à}{{\`a}}1 {è}{{\`e}}1
 {â}{{\^a}}1 {ê}{{\^e}}1 {ç}{{\c c}}1
}


\maketitle %affichage du titre
\tableofcontents %table des matières

\chapter{Introduction}
\label{cha:Introduction}
Ce document contient le code détaillé et commenté que nous avons incorporé au projet LaTeXiLa, afin d'implémenter le mécanisme de complétion dynamique de la commande \textbf{\textbackslash{}ref}.
Pour plus de détails sur le projet, veuillez vous référer au rapport complet.

\\\textit{Remarque} : dans la suite du document, les notations suivantes seront utilisées.

\begin{description}
  \item[// ...] représente une partie de code inchangée, et donc non explicitée.
  \item[\{// ...\}] représente le corps non détaillé d'une méthode, car trivial.
\end{description}

\chapter{Classe : CompletionProvider}
\label{cha:classe_completionProvider}
La classe CompletionProvider est le gestionnaire de complétion codé pour le projet LaTeXiLa.
Au début de notre étude, il ne proposait pas de complétion pour la commande \textbf{\textbackslash{}ref}, mais uniquement une complétion statique basée sur un fichier xml.
Afin d'implémenter notre solution, nous y avons apporté les modifications suivantes.

\section{Structure de Donnée}
\label{sec:CP_structure_de_données}

\begin{lstlisting}[frame=single]
public class CompletionProvider : GLib.Object, SourceCompletionProvider
{
    // Structure de donnée représentant l'arborescence
    // des choix de complétion.
    // La commande \ref a pour format : \ref{nom_du_label}
    
    // Structure de la commande
    struct CompletionCommand
    {
        string name;    // "\ref"
        string? package;
        CompletionArgument[] args;  // Un seul argument
    }
    // Structure d'argument de commande
    struct CompletionArgument
    {
        string label;
        bool optional;
        CompletionChoice[] choices;
        // L'ensemble des labels déclarés par l'utilisateur.
    }
    // Structure de la proposition de complétion
    public struct CompletionChoice
    {
        string name;    // L'intitulé du label
        string? package;
        string? insert;
        string? insert_after;
    }
    
    // Ensemble des commandes proposant des choix de complétion.
    // Doit être mis à jour pour la commande \ref.
    private Gee.HashMap<string, CompletionCommand?> _commands;
    
    // Attributs ajoutés à la classe :
    
    // Table de hachage contenant des ensembles non redondants
    // de choix de complétion d'un fichier, indexée par
    // le chemin absolu du fichier.
    // L'ajout de données à cette table de hachage est fait
    // par les instances de la classe Document.
    private Gee.HashMap<string, Gee.HashSet<CompletionChoice?>> 
      _labels_from_files =
      new Gee.HashMap<string, Gee.HashSet<CompletionChoice?>>();
    
    // Booléen utilisé pour mettre à jour les choix de
    // complétion uniquement quand cela est nécessaire
    // (soucis d'efficacité).
    private bool _labels_modified = false;
    
    // Chaîne de caractères utilisée pour filtrer les choix de
    // complétion proposés selon le contexte.
    // Le contexte ici utilisé est le chemin absolu
    // du répertoire parent du document courant.
    // On propose ainsi une complétion des labels déclarés dans
    // les fichiers .tex de ce répertoire.
    private string _last_dir = "";
\end{lstlisting}

\section{Méthodes}
\label{sec:CP_méthodes}

\begin{lstlisting}[frame=single]
public class CompletionProvider : GLib.Object, SourceCompletionProvider
{
    // Getters et setters pour certains des attributs
    // mentionnés plus haut.
    public void set_labels_modified (bool b){// ...}
    public void set_last_dir (string dir){// ...}
    // Utilisée par les instances de la classe Document.
    public Gee.HashMap<string, Gee.HashSet<CompletionChoice?>> 
      get_labels_from_files (){// ...}
    
    // Retourne un tableau de CompletionChoice correspondant
    // aux labels contenus dans la table de hachage, provenant
    // des fichiers du répertoire passé en paramètre.
    public CompletionChoice[] get_all_labels (string dir)
    {
        CompletionChoice[] choices = {};
        
        foreach (var entry in _labels_from_files.entries)
            // Filtrage des entrées selon le chemin absolu
            // du répertoire.
            if (entry.key.has_prefix (dir))
                foreach (CompletionChoice c in entry.value)
                    choices += c;
        
        return choices;
    }
    
    // Met à jour la structure des choix de complétion
    // pour la commande \ref en remplacant le champ adéquat de
    // _commands par la nouvelle liste.
    public void update_label_completion_choices ()
    {
        if (_last_dir != "")
        {
            // Liste des labels, filtrés selon le contexte.
            CompletionChoice[] choices =
              get_all_labels (_last_dir);
            CompletionCommand cmd_ref = _commands["\\ref"];

            cmd_ref.args[0].choices = choices;
            _commands["\\ref"] = cmd_ref;

            // Précise la cohérence entre les choix maintenant
            // proposés, et les données de la table de hachage.
            set_labels_modified (false);
        }
    }
    
    // Appellée lors d'une demande d'affichage des
    // choix de complétion par l'utilisateur.
    // C'est à ce moment que les choix de complétion proposés
    // sont mis à jour, si la mise à jour est requise.
    public void populate (SourceCompletionContext context)
    {
        // Si les choix proposés ne sont pas à jour avec
        // les entrées de la table de hachage.
        if (_labels_modified)
            update_label_completion_choices ();
        
        // ...
    }
}
\end{lstlisting}

En résumé, les choix de complétion pour la commande \textbf{\textbackslash{}ref} sont mis à jour, si besoin, lors de l'appel de l'utilisateur au gestionnaire de complétion.
La mise à jour consiste à filtrer l'ensemble des labels contenus dans la table de hachage selon un certain contexte, et à remplacer les choix proposés par ces nouveaux choix.
L'ajout de données dans la table de hachage est fait par les instances de la classe Document, quand nécessaire (voir chapitre suivant).

\chapter{Classe : Document}
\label{cha:classe_document}

La classe Document représente les documents ouverts dans LaTeXiLa.
Afin d'implémenter notre solution, nous y avons apporté les modifications suivantes.
\\\textit{Remarque} : Le code du parseur du contenu d'un document, permettant de détecter la déclaration des labels, n'est pas explicité. En effet, il s'agit du même parseur que celui utilisé par les instances de la classe DocumentStructure, allégé pour nos besoins. Les appels importants tels que l'ajout d'un label, sont néanmoins mentionnés.

\section{Structure de Donnée}
\label{sec:D_structure_de_données}

\begin{lstlisting}[frame=single]
public class Document : Gtk.SourceBuffer
{
    // Attribut déjà présent, représentant le fichier associé
    // à cette instance de Document.
    public File location { get; set; }

    // Référence vers le gestionnaire de complétion par défaut,
    // partagée par toutes les instances de Document.
    private CompletionProvider provider = CompletionProvider.get_default();
    
    // HashSet. Ensemble non redondant des labels déclarés
    // dans le document.
    private Gee.HashSet<CompletionProvider.CompletionChoice?> 
      _label_completion_choices =
      new Gee.HashSet<CompletionProvider.CompletionChoice?>();
}
\end{lstlisting}

\section{Méthodes}
\label{sec:D_méthodes}

\subsection{Gestion des labels}
\label{ssec:gestion_des_labels}

\begin{lstlisting}[frame=single]
public class Document : Gtk.SourceBuffer
{
    // Ajout d'un choix de complétion associé à l'intitulé
    // passé en paramètre dans le HashSet.
    public void add_label_completion_choice (string content)
    {
            CompletionProvider.CompletionChoice c = 
              CompletionProvider.CompletionChoice ();
            c.name = content;
            _label_completion_choices.add (c);
    }
    
    // Remise à zéro des choix de complétion associés
    // au document.
    // Utilisée au début de chaque parcours du document.
    public void drop_label_completion_choices ()
    {
        _label_completion_choices.clear ();
    }
    
    // Met à jour la table de hachage du gestionnaire de
    // complétion, avec comme clé le chemin absolu du
    // document, et comme valeur le HashSet maintenant complet.
    public void update_label_completion_choices_from_file ()
    {
        string file_path = location.get_parse_name ();
        
        // S'il s'agit de la création d'une entrée dans
        // la table de hachage, ie : s'il s'agit du premier
        // parcours de ce document.
        if (! already_parsed (file_path))
        {
            provider.get_labels_from_files ().@set
              (file_path, _label_completion_choices);
        }
        // S'il s'agit de la mise à jour d'une entrée dans la
        // table de hachage, ie : s'il ne s'agit pas du
        // premier parcours de ce document.
        else
        {
            provider.get_labels_from_files ()[file_path] =
              _label_completion_choices;
        }
    }
        
    // Permet de savoir si le document a déjà été parcouru,
    // ie : il existe une entrée pour ce document dans la
    // table de hachage.
    public bool already_parsed (string file_path)
    {
        return provider.get_labels_from_files ()
          .has_key (file_path);
    }
    
    // Informe le gestionnaire de complétion que des choix de
    // complétion ont été mis à jour.
    // Affecte à l'attribut _last_dir le chemin absolu
    // du répertoire parent du document, permettant ainsi le
    // filtrage des choix de complétion proposés à
    // l'utilisateur.
    public void notify_label_changed ()
    {
        provider.set_labels_modified (true);
        provider.set_last_dir (find_directory ());
    }
    
    // Retourne une chaîne de caractères représentant le chemin
    // absolu du répertoire parent du document.
    public string find_directory (){//...}
}
\end{lstlisting}

\subsection{Parseur}
\label{ssec:parseur}

\begin{lstlisting}[frame=single]
public class Document : Gtk.SourceBuffer
{
    // Méthode appellée pour demander un parcours du document.
    public void parse ()
    {
        // ...
        // Déjà existant. Le parseur demande un appel à
        // parse_impl().
        Idle.add (() =>
        {
            return parse_impl ();
        });
    }
    
    // Déjà existant. Parcours du document.
    private bool parse_impl ()
    {
        // Remise à zéro des choix de complétion associés à
        // ce document.
        drop_label_completion_choices ();
        // Informe le gestionnaire de complétion que les choix
        // de complétion associés à ce document ont été
        // modifiés.
        notify_label_changed ();
        
        // ...
        // Met à jour le contenu de la table de hachage du
        // gestionnaire de complétion.
        update_label_completion_choices_from_file ();
        // ...
    }
    
    // Déjà existant. Appellée pour détecter la présence d'un
    // élément d'un certain type.
    private bool search_markup (// ...)
    {
        type = null;
        
        // ...
        // Ajout de la gestion du cas de l'élément de type
        // label.
        if (type == StructType.LABEL) {
            // Ajout du choix de complétion associé à
            // l'intitulé du label (contents) dans le
            // HashSet.
            add_label_completion_choice (contents);
        }
        // ...
    }
}
\end{lstlisting}

\subsection{Mise à jour des labels}
\label{ssec:maj_des_labels}

\begin{lstlisting}[frame=single]
public class Document : Gtk.SourceBuffer
{
    // Méthode appellée pour charger le contenu d'un document.
    // Le paramètre booléen parse_related à été ajouté afin de
    // spécifier s'il faut analyser les documents .tex du même
    // répertoire (true), ou seulement le document
    // spécifié (false).
    public void load (File location, bool parse_related)
    {
        // ...
        if(parse_related) {
            parse_related_documents ();
        } else {
            parse ();
        }
    }
    
    // Analyse en arrière plan des documents .tex du même
    // répertoire que le document courant.
    public void parse_related_documents () {
        
        File dir = location.get_parent ();
        
        // Parcours sur les fichiers du répertoire.
        try {
            FileEnumerator enumerator = dir.enumerate_children
            (
                "standard::*",
                FileQueryInfoFlags.NOFOLLOW_SYMLINKS, 
                null
            );
            // Utilisé pour stocker les informations relatives
            // à chaque fichier trouvé.
            FileInfo info = null;
                
            while (((info = enumerator.next_file (null))
                    != null))
            {
                Document doc = new Document ();
                File child = enumerator.get_child (info);
                
                string file_path = child.get_parse_name ();

                // Si le fichier trouvé est un fichier .tex, et
                // qu'il n'a pas déjà été analysé.
                if (file_path.has_suffix (".tex") && 
                    (! already_parsed (file_path)))
                    // Analyse de ce fichier uniquement
                    // (passage de false en paramètre)
                    doc.load (child, false);
            }
        } catch (Error e) {
            warning ("%s", e.message);
        }
    }

    // Méthode appellée lors de la sauvegarde du document.
    // Nous lui avons ajouté une demande de parcours du
    // document, afin de mettre à jour l'ensemble des
    // choix de complétion de la table de hachage
    // associée à ce document.
    public void save (// ...)
    {
        // ...
        parse();
    }
}
\end{lstlisting}

\subsubsection{Les appels à load}
\label{sssec:les_appels_à_load}
Ayant modifié le prototype de la méthode \textit{load} en lui ajoutant un paramètre booléen, nous avons du donner une valeur à ce paramètre dans d'autres classes, qui faisaient appel à cette méthode.
Plus précisément, c'est grâce à cette différenciation lors de l'appel que l'on a pu demander l'analyse des document .tex du même répertoire lors de \textit{l'ouverture} d'un document, et le changement de contexte de complétion lors du \textit{changement de document courant} parmis les documents ouverts.
Voici les différents appels modifiés.

\begin{lstlisting}[frame=single]
public class DocumentTab : Grid
{
    // Appellée lors de l'ouverture d'un document .tex.
    public DocumentTab.from_location (File location)
    {
        this ();
        // Demande d'analyse des documents .tex du même
        // répertoire.
        document.load (location, true);
    }
    
    // Lors du changement de document courant.
    private bool view_focused_in ()
    {
        // ...
        // Si le document à été modifié par un programme tiers,
        // il faut de nouveau l'analyser.
        if (document.is_externally_modified ())
        {
            // ...
            // On analyse le document.
            // On fait ici le choix d'analyser aussi les
            // documents du même répertoire, car rien
            // n'indique qu'ils n'ont pas aussi été modifiés.
            // Cette "mesure de sécurité" est en réalité
            // indispensable.
            document.load (document.location, true);
            // ...
        }

        // Dans tous les cas, on change le contexte de
        // complétion. On demande donc la mise à jour
        // des choix de complétion proposés, en prenant
        // en compte le répertoire parent du nouveau
        // document courant.
        document.notify_label_changed ();
        // ...
    }
}
\end{lstlisting}

En résumé, un document peut être analysé dans deux cas : s'il s'agit de l'ouverture du document ou s'il a été modifié par l'extérieur, l'analyse les documents .tex du même répertoire est nécessaire.
Dans ce cas, on effectue un parcours sur les fichiers .tex du même répertoire, et on met à jour les entrées de la table de hachage du gestionnaire de complétion.
\\Au contraire, s'il s'agit d'un changement de document courant, on change simplement le contexte de complétion, forcant ainsi la mise à jour des choix de complétion proposés à l'utilisateur si ce dernier fait appel au gestionnaire de complétion.
\\Le parcours d'un document est également fait à chaque sauvegarde, permettant ainsi de toujours garder à jour les entrées de la table de hachage. Lors du prochain appel au gestionnaire de complétion, les choix proposés seront de nouveau filtrés, et seront donc pertinents.
\end{document}
