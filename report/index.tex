\documentclass[a4paper,11pt]{report}
\usepackage[T1]{fontenc}
\usepackage[utf8]{inputenc}
\usepackage{lmodern}
%\usepackage{exptech}

\title{Étude pratique : Amélioration de la complétion automatique de LaTeXila}
\author{Axel Caro\and François Bochet\and Maximilien Richer}
\date{2014-2015}

\begin{document}

\maketitle %affichage du titre
\tableofcontents %table des matières

\section{Remerciments} % (fold)
\label{cha:remerciments}
Nous remercions Arnaud Blouin, notre encadrant, pour sa disponibilité et ses conseils, ainsi que Sébastien Wilmet, développeur et mainteneur de LaTeXila.

\section{Introduction}
\label{cha:Introduction}
Cette étude pratique a été pour nous l'occasion de découvrir le fonctionement des projets open-source.

\section{Latexila}
\label{sec:latexila}

Latexila est un projet d'éditeur LaTeX pour le projet GNOME

\section{Compiler le projet}
\label{sec:compiler}
La compilation d'un projet GNOME est un processus complexe pour un non-initié. En effets, elle demande certaines conaissances dans le fonctionement des gestionaires de paquet ansi que des modes de focntionement des distributions Linux modernes.

\subsection{Jhbuild}
\label{sub:jhbuild}

\section{Etude pratique} % (fold)
\label{cha:etude_pratique}

\subsection{Gestion du projet} % (fold)
\label{sub:git}
Git est le CVS utilisé par la très grande majorité des projets open-source. LaTeXiLa est hébergé sur le dépôt du projet Gnome, mais un mirroir est diponible sur Github %adresse 
Pour ce pojet, l'équipe a crée un \textit{fork} du dépot principale sur Github et a travaillé sur une branche annexe, régulièrement mise à jour à partir du dépot officiel.
Les soumissions sont faites via le bugzilla du projet Gnome, qui permet au mainteneur de commenter le code soumis.
% subsection git (end)

\section{Tâche à réaliser} % (fold)
\label{sec:tache_a_realiser}

\section{Réalisation}

% section tâche_à_réaliser (end)

\end{document}
